\chapter*{附录}
\appendix
\renewcommand{\appendixname}{Appendix~\Alph{section}}
\section*{致谢}

\quad 我们对以下团体或个人表示由衷的感谢,没有他们,马家沟河项目是无法走到今天的。

\begin{itemize}
    \item 感谢龙芯中科提供的设备与比赛平台支持,衷心祝愿他们能够在国产CPU道路上越走越远。
    \item 感谢舒燕君和刘国军两位指导老师的帮助、指导和关心。
    \item 感谢胡光辉等隔壁团队同学们的无私帮助。
    \item 感谢张清钰、王永琪、郑翔宇、丛日东等学长的帮助。
    \item 感谢清华大学ZenCove、NonTrivalMips、重庆大学CDIM等往届团队,他们开源的往届作品及经验分享对我们的工作起到了极大帮助和启发作用。
    \item 感谢姚永斌老师的《超标量处理器设计》一书,此书给予了我们很大启发,没有它就没有Ma-River现在的微架构设计。
\end{itemize}

\section*{参考资料}

\quad 本项目参考了包括但不限于下列书籍、资料、网站或开源项目:

\begin{itemize}
    \item 姚永斌. 超标量处理器设计[M]. 清华大学出版社: 201404.
    \item 唐朔飞. 计算机组成原理(第三版)[M]. 哈尔滨工业大学出版社: 202010.
    \item 汪文祥,邢金璋. CPU设计实战[M]. 机械工业出版社: 202101.
    \item CEMU文档:\url{http://cemu.cyyself.name/}
    \item CDIM项目:\url{https://github.com/Maxpicca-Li/CDIM}
    \item ZenCove项目:\url{https://github.com/zencove-thu/}
    \item NonTrivalMips项目:\url{https://github.com/trivialmips/nontrivial-mips}
    \item SHIT Core项目:\url{https://github.com/Superscalar-HIT-Core/Superscalar-HIT-Core-NSCSCC2020}
    \item TinyVGA:\url{http://tinyvga.com/}
    \item Framebuffer\_driver\_rpi:\url{https://github.com/dsoastro/framebuffer_driver_rpi}
\end{itemize}