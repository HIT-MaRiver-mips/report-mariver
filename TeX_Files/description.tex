\chapter{概述}

\section{项目背景}

马家沟河(Ma-River)是第七届”龙芯杯“全国大学生计算机系统能力培养大赛(NSCSCC 2023)的参赛作品。

项目实现了基于MIPS32 Release1指令集的多发射乱序处理器软核——Ma-River CPU,并在此基础上自行构建了SoC,自行设计实现了多个外设控制器,可以运行U-Boot、ucore、Linux,并为其移植或自行实现了大量丰富应用,实现了一个完整的基于FPGA的嵌入式计算机系统。

\section{项目概述}

\subsection{设计模式}

Ma-River的所有硬件设计使用Verilog HDL实现。

在项目前期的结构设计阶段,我们使用C++编写了模拟CPU结构的周期精确模拟器,根据其在benchmark上的性能表现,进行精确的结构调整,使我们以极低代价得到具有优秀IPC的精简结构设计方案。后期直接据此进行RTL实现。

在项目后期的调试阶段,我们自行实现了指令自动生成器,可以根据不同场景生成大量随机指令,通过trace比对尽可能消除CPU核的潜在bug。

\subsection{CPU架构}

Ma-River CPU是六发射超标量乱序处理器,支持128条MIPS32 Release1指令。主流水线11级,取指/提交宽度为2,后端具有用于指令执行的4个整数功能单元和2个浮点功能单元,每周期最多发射执行6条指令。处理器核在整体上分为整数和浮点两个相对独立的部分,二者共享取指/访存/ROB/提交部件。

Ma-River CPU正确实现了CP0、TLB、精确异常等系统支持所需部件,能够支持Linux等操作系统的正常启动运行。

\subsection{SoC设计}
为了充分利用实验箱上的提供的丰富外设、发挥CPU核的性能,我们基于Vivado Block Design设计了一套功能较完备的SoC。我们通过自行设计的控制器和已有的IP核,能够操作板上的绝大多数功能模块,包括:

- VGA:控制器自行实现;分辨率1024*768,支持字符和图像模式;支持DMA,适配了Framebuffer驱动

- LCD:控制器自行实现;分辨率480*800,支持字符和图像模式;支持DMA,适配了Framebuffer驱动

- LCD触摸:使用AXI IIC IP核实现,并使用自行实现的模块进行硬复位

- PS/2:控制器自行实现;支持PS/2键盘输入

- GPIO:控制器自行实现;支持以中断方式获取按键状态,支持控制LED、数码管

- DRAM:使用Xilinx MIG 7 Series IP核实现

- 串口:使用AXI UART16550 IP核实现

- 以太网:使用AXI Ethernetlite IP核实现

同时,我们通过拓展I/O接口,可对板外SD卡进行读写操作。


\subsection{系统软件}

Ma-River使用U-Boot进行引导,可以启动ucore和Linux(最新的6.4版本)操作系统。在Linux上,我们自行实现和移植了VGA Console、LCD、FrameBuffer、键盘等驱动程序,并且还实现了Linux终端的中文输入输出支持。

我们为Linux移植了buildroot,通过网口挂载NFS文件系统,并在此基础上移植了GCC、Python、QEMU等软件,自行编写或移植了LVGL GUI、Gif动画播放器、3D实时渲染器等图形应用。